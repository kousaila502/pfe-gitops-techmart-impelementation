\thispagestyle{empty}

\begin{center}
{\Large \textbf{Résumé}}
\end{center}

\vspace{1cm}

Cette recherche aborde la lacune critique dans l'évaluation empirique des méthodologies de déploiement modernes en menant une comparaison complète entre les approches GitOps et CI/CD traditionnelles à travers une implémentation et une analyse pratiques.

L'étude implémente TechMart, une plateforme e-commerce de niveau production conçue comme une architecture de microservices multi-cloud pour servir à la fois d'application fonctionnelle et d'environnement de recherche contrôlé. La plateforme comprend quatre microservices distincts utilisant diverses piles technologiques incluant Python FastAPI, Node.js Express, Java Spring Boot, et plusieurs technologies de bases de données, déployés sur l'infrastructure Google Kubernetes Engine et Heroku.

La méthodologie de recherche emploie une approche systématique en deux phases conçue pour établir les caractéristiques de base et permettre une comparaison équitable des méthodologies à travers des architectures de services hétérogènes. L'implémentation démontre des modèles de déploiement pratiques pour les méthodologies GitOps et CI/CD traditionnelles tout en abordant la complexité du monde réel et les contraintes opérationnelles.

L'implémentation GitOps utilise ArgoCD pour la gestion d'infrastructure déclarative et le déploiement continu, démontrant la synchronisation automatisée, les capacités d'auto-guérison, et la gestion de configuration basée sur Git. L'implémentation CI/CD traditionnelle démontre les modèles de déploiement conventionnels à travers les workflows GitHub Actions avec optimisation spécifique à la plateforme et mécanismes de surveillance opérationnelle.

La conception architecturale incorpore des cadres de surveillance et d'observabilité complets permettant une mesure de performance détaillée et une analyse opérationnelle. Un cadre de normalisation de complexité novateur est développé pour permettre une comparaison équitable à travers différents types de services et piles technologiques en tenant compte de divers facteurs de complexité incluant les caractéristiques du code source, les exigences de construction, et les dépendances opérationnelles.

L'étude valide la faisabilité d'architecture hybride, démontrant une intégration transparente entre les méthodologies GitOps et CI/CD traditionnelles au sein du même écosystème d'application. Cette approche permet des stratégies de migration pratiques et une application sélective de méthodologie basée sur les caractéristiques des services et les exigences organisationnelles.

La recherche contribue à la connaissance en ingénierie logicielle à travers l'innovation méthodologique dans l'évaluation des méthodologies de déploiement, les modèles d'implémentation pratiques pour les architectures de microservices multi-cloud, et les cadres basés sur l'évidence pour la prise de décision technologique d'entreprise.

\textbf{Mots-clés :} GitOps, CI/CD Traditionnel, Architecture Multi-Cloud, Microservices, Méthodologies de Déploiement, DevOps, Plateforme TechMart, ArgoCD

\newpage

\newpage