\thispagestyle{empty}

\begin{center}
{\Large \textbf{\textarabic{ملخص}}}
\end{center}

\vspace{1cm}

\begin{otherlanguage}{arabic}
\begin{RTL}

يقدم هذا البحث أول مقارنة تجريبية شاملة بين منهجيات GitOps و CI/CD التقليدية باستخدام منصة خدمات مصغرة متعددة السحابات بمستوى الإنتاج. من خلال تنفيذ TechMart، وهي منصة تجارة إلكترونية وظيفية تمتد عبر بنية Google Kubernetes Engine و Heroku، تعالج هذه الدراسة الفجوات الحاسمة في تقييم منهجيات النشر من خلال التحليل الإحصائي الصارم وتطبيع التعقيد.

يستخدم التحقيق نهجاً منتظماً من مرحلتين يشمل 47 تجربة مضبوطة مع التحقق من الدلالة الإحصائية (p < 0.01). تضع المرحلة الأولى خصائص المنهجية الأساسية من خلال مقارنة الخدمة الواحدة، بينما تنفذ المرحلة الثانية تطبيع التعقيد متعدد الخدمات مما يتيح التقييم العادل عبر المكدسات التكنولوجية المتجانسة بما في ذلك تطبيقات Python FastAPI و Node.js Express و Java Spring Boot.

تكشف النتائج الرئيسية عن مقايضات أساسية بين منهجيات النشر: يُظهر CI/CD التقليدي أداء بناء متفوق بـ 2.3 مرة (57 ثانية مقابل 132.5 ثانية في المتوسط)، بينما يحقق GitOps 100% أتمتة مع استرداد فشل أسرع بـ 17 مرة (23-37 ثانية تلقائي مقابل 5-15 دقيقة يدوي).

يطور البحث إطار عمل تطبيع التعقيد الجديد الذي يتيح المقارنة العادلة عبر هياكل الخدمات المتجانسة من خلال مراعاة تعقيد قاعدة الكود (20%)، ومتطلبات البناء (25%)، وكثافة الموارد (20%)، وخصائص المكدس التكنولوجي (15%)، والتبعيات الخارجية (10%)، وتعقيد هدف النشر (10%).

توصي أطر اتخاذ القرار المؤسسي المستمدة من الأدلة التجريبية بـ CI/CD التقليدي للفرق الصغيرة (<10 مطورين) التي تعطي الأولوية لسرعة البناء، والهياكل الهجينة للفرق المتوسطة (10-50 مطوراً) التي تتيح التبني التدريجي، و GitOps للفرق الكبيرة (50+ مطوراً) التي تتطلب قابلية التوسع التشغيلي وتميز الأتمتة.

يضع البحث معايير جديدة لتقييم منهجية CI/CD من خلال التحليل التجريبي بمستوى الإنتاج، مما يوفر رؤى قائمة على الأدلة لقرارات الاستثمار التكنولوجي المؤسسي بينما يحدد مسارات التحسين الملموسة لكلا النهجين المنهجيين.

\textbf{الكلمات المفتاحية:} GitOps، CI/CD التقليدي، هندسة متعددة السحابات، الخدمات المصغرة، تحليل الأداء، DevOps، دراسة تجريبية، منصة TechMart

\end{RTL}
\end{otherlanguage}

\newpage