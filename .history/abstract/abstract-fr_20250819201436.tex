\thispagestyle{empty}

\begin{center}
{\Large \textbf{Résumé}}
\end{center}

\vspace{1cm}

Cette recherche présente la première comparaison empirique complète des méthodologies GitOps et CI/CD traditionnelles utilisant une plateforme de microservices multi-cloud de niveau production. À travers l'implémentation de TechMart, une plateforme e-commerce fonctionnelle s'étendant sur l'infrastructure Google Kubernetes Engine et Heroku, cette étude aborde les lacunes critiques dans l'évaluation des méthodologies de déploiement grâce à une analyse statistique rigoureuse et une normalisation de la complexité.

L'investigation emploie une approche systématique en deux phases comprenant 47 expériences contrôlées avec validation de signification statistique (p < 0,01). La Phase 1 établit les caractéristiques de base des méthodologies à travers une comparaison mono-service, tandis que la Phase 2 implémente une normalisation de complexité multi-services permettant une évaluation équitable à travers des piles technologiques hétérogènes incluant Python FastAPI, Node.js Express, et Java Spring Boot.

Les résultats clés révèlent des compromis fondamentaux entre les méthodologies de déploiement : le CI/CD traditionnel démontre une performance de construction 2,3x supérieure (57s vs. 132,5s en moyenne), tandis que GitOps atteint 100\% d'automatisation avec une récupération de panne 17x plus rapide (23-37s automatique vs. 5-15 minutes manuel). L'analyse d'attribution de performance critique identifie la configuration du service d'authentification comme contribuant à 65\% des différences de performance, fournissant des opportunités d'optimisation immédiates indépendamment de la sélection de méthodologie.

La recherche valide l'intégration d'architecture hybride à surcharge zéro, première dans l'industrie, démontrant une coexistence transparente de GitOps et CI/CD traditionnel avec confirmation statistique d'aucune pénalité de performance mesurable (p > 0,05). Cette découverte permet des stratégies de migration pratiques et une application sélective de méthodologie basée sur les caractéristiques des services plutôt que sur les contraintes architecturales.

L'étude développe un cadre de normalisation de complexité novateur permettant une comparaison équitable à travers des architectures de services hétérogènes en tenant compte de la complexité du code source (20\%), des exigences de construction (25\%), de l'intensité des ressources (20\%), des caractéristiques de la pile technologique (15\%), des dépendances externes (10\%), et de la complexité de la cible de déploiement (10\%).

Les cadres de décision d'entreprise dérivés des preuves empiriques recommandent le CI/CD traditionnel pour les petites équipes (<10 développeurs) priorisant la vitesse de construction, les architectures hybrides pour les équipes moyennes (10-50 développeurs) permettant une adoption graduelle, et GitOps pour les grandes équipes (50+ développeurs) nécessitant une évolutivité opérationnelle et une excellence d'automatisation.

\textbf{Mots-clés :} GitOps, CI/CD Traditionnel, Architecture Multi-Cloud, Microservices, Analyse de Performance, DevOps, Étude Empirique, Plateforme TechMart

\newpage