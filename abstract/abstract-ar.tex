\thispagestyle{empty}
\vspace*{\fill}

\begin{center}
{\Large \textbf{\textarabic{ملخص}}}
\end{center}

\vspace{1cm}

\begin{otherlanguage}{arabic}
\begin{RTL}
يقارن هذا البحث بين طريقتين شائعتين لنشر تطبيقات البرمجيات: GitOps والطريقة التقليدية CI/CD. الهدف هو فهم أي طريقة تعمل بشكل أفضل في حالات مختلفة.

لإجراء هذه المقارنة، قمنا ببناء TechMart، وهو موقع تجارة إلكترونية حقيقي يحتوي على أربع خدمات منفصلة (إدارة المستخدمين، كتالوج المنتجات، عربة التسوق، ومعالجة الطلبات). كل خدمة تستخدم لغات برمجة مختلفة مثل Python و Node.js و Java، وتخزن البيانات في قواعد بيانات مختلفة.

تعمل المنصة على عدة موفري خدمات سحابية - بعض الخدمات على Google Cloud وأخرى على Heroku. هذا الإعداد يسمح لنا بتجربة كلا طريقتي النشر في ظروف واقعية.

يستخدم GitOps أداة تسمى ArgoCD التي تنشر التطبيقات تلقائياً من خلال مراقبة تغييرات الكود في مستودعات Git. عندما يحدث المطورون الكود، يطبق النظام هذه التغييرات تلقائياً دون تدخل بشري.

يستخدم CI/CD التقليدي GitHub Actions لبناء ونشر التطبيقات خطوة بخطوة، وغالباً ما يتطلب موافقة يدوية قبل النشر للإنتاج.

صممنا نهج اختبار منتظم لمقارنة كلا الطريقتين بعدالة. يشمل ذلك قياس أوقات البناء وسرعة النشر ومدى جودة تعامل كل طريقة مع الأخطاء.

تظهر الدراسة أن كلا الطريقتين لهما نقاط قوة. GitOps يوفر أتمتة أفضل واستعادة أسرع من الأخطاء، بينما CI/CD التقليدي يقدم أوقات بناء أسرع وإعداد أبسط.

الأهم من ذلك، أثبتنا أن كلا الطريقتين يمكن أن تعملا معاً في نفس التطبيق. هذا يعني أن الشركات لا تحتاج لاختيار طريقة واحدة فقط - يمكنها استخدام أفضل طريقة لكل خدمة حسب احتياجاتها المحددة.

يساعد هذا البحث فرق البرمجيات على اتخاذ قرارات أفضل حول أي طريقة نشر تستخدم لمشاريعها.

\textbf{الكلمات المفتاحية:} GitOps، CI/CD التقليدي، نشر البرمجيات، الحوسبة السحابية، الخدمات المصغرة، TechMart، ArgoCD
\end{RTL}
\end{otherlanguage}

\vspace*{\fill}
\newpage